\chapter{Evaluation}
\section{Success of solution}
Following both my alpha and beta testing conducted by the project stakeholders and I, the website can be seen to achieve the goals which it was designed to. The auction website provided the core functionality required: being able to create a listing and being able to bid on listings. This alongside its additional features of messaging and price recommendation mean that it can be considered a success.
The first part of the success criteria was designed to ensure that the user could login and register for the website. This was met and exceeded through the design of a simple and secure login system that utilised the OpenSSL encryption framework \parencite{openssl} in order to keep users details safe in the event of a cyber-attack on the database. The login and register systems were simple through a single form on a single page that collected all of the user’s details that were required. This was also a good design as ensuring that information such as the user’s username were collected meant that the site could run and function smoothly. 
The homepage’s aim was to be a clean page whereby the user could access any part of the site they wished to. During my preliminary research of already existing design solutions, I found that eBay \parencite{ebay1} and especially the London Camera Exchange \parencite{lcegroup} both had complex and difficult to navigate homepages which severely detracted from the user-friendliness of the site. My homepage design met its aims and provided a modern interface whereby the user could easily access the function they wanted through the use of large buttons and a navigation bar. The homepage also contained the code to remove listings once they had expired, this was beneficial as it meant that whenever the user went to the homepage on the website, the listings would be removed that were out of time. This was in order to ensure that the website was as up to date as possible allowing me to better meet the brief and make buying and selling listings easier.
Creating a listing is one of the most important parts of the website. I felt that the create a listing page was good yet had some flaws in the design. Whilst the page did achieve the desired functionality as it allowed the user to create a detailed listing include an image. However, my success criteria was that the user would be able to add a quality indicator and my wireframe screen design in Figure 9 showed that the user would be able to enter multiple images for a listing. Neither of these two concepts were implemented due to time constraints on the project. The page’s design did allow it to meet the task of encouraging the user to be descriptive about the item. This was with the aim of improving the overall buying process for the customer.
The search feature was the key part of the website that allowed the user to discover the listings that were available on the site. The search feature could either be accessed through the conventional page whereby the user would be shown a large search box which would return the results. Alternatively, the navigation bar contained a search box that once submitted would also show the results. If I was to develop the program further, I would look to add filtering and sorting to the search feature which would allow the user to have a better experience and be able to find the camera that they wanted easier. 
If the user chose to view a specific listing for the page, they would be shown the specific information for the listing. Part of the success criteria was that the user would be able to bid on a listing if they wanted to. This feature of the site was well implemented with the user being able to bid on the item and see it updated. This required adding validation in order to check that the bid was not smaller than the one that was already entered.
Parts of the success criteria have not been able to be met. One such page was the camera information page which was designed to provide all the technical details of the camera which the user searched for. This feature was not built either as a standalone page nor as a part of the view listing. Whilst the information does exist in the database, it failed to be implemented due to the amount of code needed to load each row from the table. In early testing, I also found that the listings page would load considerably slower even with limited amount of information loaded. In order to continue to meet the brief of a fast and easy website, I decided to not implement the page and leave it for further development. 
The full listing ending criteria has not been met due to time and development constraints. Whilst the homepage meant that listings that have ended are removed from the site, the taking of payment and the emailing the user have not been built. Due to the nature of the project development, mainly single student with limited resources, I decided not include the payments system. It wasn’t possible to sort multiple bank accounts for testing and to find people willing to conduct full testing. I was also unable to implement the end of listing email to both the seller and the buyer. This was due to the email system only being built late into the project development. My primary aim for the email function was in use for a password reset system. This implementation look longer than expected and meant that I did not have the time required to create the desired email functions when I outlined the project. 
Part of the success criteria was that the sold listings information would be saved when the listing has ended. This was with the idea of creating a price recommendation feature on the website to better assist the buying and selling of camera gear. The sold listings details were able to be saved during the removal process that takes place from the home page. The updating of a recommended price was later added as a feature of its own standalone page with the relevant scripts running from that page. This helped the user to list cameras at a better price point for their item which would help both entice customers but also help create a quick sale for the item. 

\section{Final solution}
\subsection{Product description}
The project has gone above what original features were outlined for the website. One such feature was the ability for the project to give a price recommendation for the user. This feature meant would allow the user to take the sold listings from the website and use them to get a recommendation for a price for the camera of their choices based on what had sold. The feature uses all the entries in the sold listings in the database and average out a price on a per camera basis before storing it in a separate database table. The python script would run each time the user entered the page which meant that records were always up to date. This feature was well approved by the stakeholders both in the concept phase and in the final implementation with both feeling it helped to improve the selling and buying process.
In order to keep the user and their details safe with the website, I implemented an advanced form of encryption in order to ensure that the users’ details are not stored in plain text and cannot be read easily. This would become a vital part of the website if it were to be publicly launched. Since large amounts of data is being stored and specifically sensitive user data in the form of card details, the website would have to become compliant with the UK data protection act. (Data Protection Act, 2018) One of the conditions being the secure storage of data so that it cannot be assessed.

Following the development of prototype 3, the decision was taken to add a forgot password feature to the website. This would allow the user to get a new password in the event that they forgot their original password. The password was a collection of randomly generated numbers and letters to ensure that a user always had a secure password. The system worked by getting the user to enter both their username and their email on a separate form situated on another page. The username is required so that the email that the user has entered can be verified against what it should be so that only the user can update their password. Providing the details match, the user is sent an email with their new password allowing them continued access to the website. 
\subsection{Further development} 
If the Camera Auction Website was to launch and go public, there would be certain elements of the website which would need to be modified, removed, or implemented in order to meet the various web standards and guidelines that are required in the 21st Century. Such standards include an SSL certificate \parencite{ssl} and adding compliance for WCAG 2 \parencite{w3c_accessibility_standards}. An SSL certificate is required for pages in order to make them show up as secure in web browsers, an important addition for the website in order to attract customers and to ensure the security of the buying and selling of user’s data. The WCAG guidelines are a set of suggestions for how websites can be adapted an improved in order to make them more accessible for disabled people when they use the website. 

WCAG can be broken down into 4 criteria for each webpage, each page must be: perceivable, operable, understandable and robust for those looking to operate the website with special requirements. When looking at the further development of the webpage, it is important to take measures in order to meet each guideline and bring each page up to the standard that is required. The search and create listing pages would both require updates to how the images are processed and displayed to meet the condition that all images contain a text description for use with voice over utilities. Whilst the website would mostly meet the operable condition of the WCAG guidelines, longer display time and clearer error and success messages would have to be added. This would be to ensure that users had enough time to read the content before it was hidden again. Part of the regulations also include the idea of making the website understandable and behave in predicable ways. This could be further improved throughout the website due to the current lack of web design that adjusts based on device size. Due to errors within the CSS, the website does not always have a responsive web design and elements are not consistently arranged which could be difficult for those with disabilities operating the website. Users are also not guided when it comes to fixing errors on the page. The styling may also need to be adjusted in order to create more consistency for the user between the registration pages and the home page. Another condition is that the website must help the user avoid and correct mistakes. Whilst each page contains a diverse set of error messages which aim to guide the user to which part of the form is causing an error, they could be more specific and could give more detail for those who need the extra support. All these adjustments should help to meet the robust condition which ensures that the website will be consistent for future iterations and technology.

An SSL certificate binds a document to a cryptographic key pair consisting of a private and public key. Its aim is to ensure the validity of a website through creating a registry of its owners and the domain owners and to make sure that the website can be trusted by browsers, operating systems and users. Most of the requirements are centred around the domain and hosting of the website but some elements of the website would have to be updated so that they create a secure connection using TLS (Transport Layer Security). This would require updates to the encryption and authentication for the website. The website would also have to be registered and attached to a web domain through services such as Fasthosts \parencite{fasthost} or NameCheap \parencite{namecheap}. These services would likely provide the hosting for the website which would include the SSL certificate and the condition required for it.
Part of the original success criteria for the project was the requirement of providing specific camera information for the model that was being sold. This would have been fetched from the database. Whilst camera information is currently stored in the database, there is currently no place where such information is displayed to the user. In order to best implement it, I would peruse a style similar to that of MPB \parencite{mpb} whereby the information is displayed at the bottom of every listing. This would mean that each listing would be clearer to the user and would show the user more specifically what they were getting with the camera that they have brought. 
The messages feature of the page built for prototype 3, could be further developed in order to provide a better experience for the user. Whilst there is an ability to send a message to the user, there is currently no way to be able to reply, delete or even mark a message as read. As I look to develop the project further, I would want to overhaul the page, redeveloping it to provide a better communication platform so that the buyer and seller can better communicate. I would also look to add an implementation for email notifications when the user receives a new message from the seller. 

As the website grows, a redevelopment of certain features will likely be necessary. Throughout development, areas such as the websites encryption began to cause issues such as different outputs depending on the page that it was using. To solve this, an object orientated programming solution would have to be created in order to make encryption and decryption functions. These could be called upon on each page with the inputs being passed as parameters and the functions return being assigned to variables. This would help to make encryption outputs consistent and decrease development time in the long run. A similar approach was used when creating an email system since all emails are sent through a single python program which takes all requirements as parameters. PHP has also reached end of life status \parencite{php_versions} for the various versions used over the course of the development of the project. To rectify this, the website may eventually have to be rewritten in a more modern programming language featuring up to date web frameworks such as Django \parencite{django}, NodeJS \parencite{nodejs} and React \parencite{meta_platforms_react}.
